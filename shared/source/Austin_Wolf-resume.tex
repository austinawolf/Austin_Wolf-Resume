\documentclass[letterpaper,10pt]{article}

\usepackage{latexsym}
\usepackage[empty]{fullpage}
\usepackage{titlesec}
\usepackage{marvosym}
\usepackage[usenames,dvipsnames]{color}
\usepackage{verbatim}
\usepackage{enumitem}
\usepackage[hidelinks]{hyperref}
\usepackage{fancyhdr}
\usepackage[english]{babel}
\usepackage{tabularx}
\input{glyphtounicode}


%----------FONT OPTIONS----------
% sans-serif
% \usepackage[sfdefault]{FiraSans}
% \usepackage[sfdefault]{roboto}
% \usepackage[sfdefault]{noto-sans}
% \usepackage[default]{sourcesanspro}

% serif
% \usepackage{CormorantGaramond}
% \usepackage{charter}


\pagestyle{fancy}
\fancyhf{} % clear all header and footer fields
\fancyfoot{}
\renewcommand{\headrulewidth}{0pt}
\renewcommand{\footrulewidth}{0pt}

% Adjust margins
\addtolength{\oddsidemargin}{-0.5in}
\addtolength{\evensidemargin}{-0.5in}
\addtolength{\textwidth}{1in}
\addtolength{\topmargin}{-.5in}
\addtolength{\textheight}{1.0in}

\urlstyle{same}

\raggedbottom
\raggedright
\setlength{\tabcolsep}{0in}

% Sections formatting
\titleformat{\section}{
  \vspace{-4pt}\scshape\raggedright\large
}{}{0em}{}[\color{black}\titlerule \vspace{-5pt}]

% Ensure that generate pdf is machine readable/ATS parsable
\pdfgentounicode=1

%-------------------------
% Custom commands
\newcommand{\summaryItem}[1]{
  \item\small{
    {#1 \vspace{-2pt}}
  }
}

\newcommand{\resumeItem}[1]{
  \item\small{
    {#1 \vspace{-2pt}}
  }
}

\newcommand{\resumeSubheading}[4]{
  \vspace{-2pt}\item
    \begin{tabular*}{0.97\textwidth}[t]{l@{\extracolsep{\fill}}r}
      \textbf{#1} & #2 \\
      \textit{\small#3} & \textit{\small #4} \\
    \end{tabular*}\vspace{-7pt}
}

\newcommand{\resumeSubSubheading}[2]{
    \item
    \begin{tabular*}{0.97\textwidth}{l@{\extracolsep{\fill}}r}
      \textit{\small#1} & \textit{\small #2} \\
    \end{tabular*}\vspace{-7pt}
}

\newcommand{\resumeProjectHeading}[2]{
    \item
    \begin{tabular*}{0.97\textwidth}{l@{\extracolsep{\fill}}r}
      \small#1 & #2 \\
    \end{tabular*}\vspace{-7pt}
}

\newcommand{\resumeSubItem}[1]{\resumeItem{#1}\vspace{-4pt}}

\renewcommand\labelitemii{$\vcenter{\hbox{\tiny$\bullet$}}$}

\newcommand{\resumeSubHeadingListStart}{\begin{itemize}[leftmargin=0.15in, label={}]}
\newcommand{\resumeSubHeadingListEnd}{\end{itemize}}
\newcommand{\resumeItemListStart}{\begin{itemize}}
\newcommand{\resumeItemListEnd}{\end{itemize}\vspace{-5pt}}

%-------------------------------------------
%%%%%%  RESUME STARTS HERE  %%%%%%%%%%%%%%%%%%%%%%%%%%%%
\begin{document}

%----------HEADING----------
\begin{center}
    \textbf{\Huge \scshape Austin Wolf} \\ \vspace{1pt}
    \small \textit{} \\
    \href{mailto:austinawolf@gmail.com}{austinawolf@gmail.com} $|$ 
    \href{https://www.linkedin.com/in/austin-wolf-b9120772/}{www.linkedin.com/in/austin-wolf-b9120772} $|$
    \href{https://github.com/austinawolf}{www.github.com/austinawolf}
\end{center}

%-----------SUMMARY-----------
\section*{Summary}
\begin{itemize}
    \summaryItem{Firmware engineer in medical device and biotechnology industry with a passion for event-driven real-time embedded software architectures and networks}
    \summaryItem{Proficient in IEC-62304 compliant medical device software development, documentation, design verification, and best practices}
    \summaryItem{Hands-on experience in debugging hard faults, race conditions, memory corruption, and other critical low level issues on ARM Cortex-A/R/M processors }
    \summaryItem{Expert in robust safety-critical embedded DevOps pipelines: build systems, version control, test-driven development, static analysis, continuous integration, deployment, hardware-in-the-loop testing}
  \end{itemize}

%-----------EDUCATION-----------
\section{Education}
  \resumeSubHeadingListStart
    \resumeSubheading
      {San Diego State University}{San Diego, CA}
      {M.S. Electrical Engineering, emphasis in Embedded Systems, GPA: 3.90 }{2016 - 2019}
    \resumeSubheading
      {Chapman University}{Orange, CA}
      {B.S. Biochemistry, Minor in Computational Science, GPA: 3.72}{2012 - 2016}
  \resumeSubHeadingListEnd

%-----------EXPERIENCE-----------
\section{Experience}
  \resumeSubHeadingListStart
    \resumeSubheading
      {Illumina}{Feb 2023 -- Current}
      {Staff Firmware Engineer}{San Diego, CA}
      \resumeItemListStart
        \resumeItem{Migrated embedded firmware on Zynq Ultrascale+ from Cortex-A53 to Cortex-R5, including platform driver porting, MPU configuration, low-level assembly initialization, custom bootloader implementation, and SEGGER J-Link support setup}
        \resumeItem{Architected a protobuf-based, RPC-style communication framework for firmware-software Ethernet messaging, significantly improving memory usage, latency, and developer usability}
        \resumeItem{Led multiple initiatives to strengthen compliance with IEC 62304 and software development lifecycle procedures, including unit test framework down-select, static analysis integration, formalized release/version control, and creation of design documentation templates}
        \resumeItem{Developed a C++ application for fiducial detection on NVIDIA Jetson Orin Nano using CUDA-accelerated OpenCV, porting a Python-based computer vision algorithm to C++ for real-time performance}
        \resumeItem{Ported the latest Eclipse ThreadX RTOS and NetX network stack STM32H7, Zynq-7000, and Zynq Ultrascale+ SoCs, enabling unified OS support across platforms}
        \resumeItem{Optimized GitHub Actions CI/release pipeline, reducing build times by 8× and resolving pain points in deployment workflow}
        \resumeItem{Collaborated with cybersecurity teams to learn firmware security best practices, develop a threat model, and create a cybersecurity FMEA for firmware on Illumina instruments}
      \resumeItemListEnd
    \resumeSubheading
      {Novo Engineering}{July 2019 -- Feb 2023}
      {Senior Firmware Engineer}{Vista, CA}
      \resumeItemListStart
        \resumeItem{Project Recell: led implementation of full closed-loop PID motor control stack on STM32F4 processor for precise robotic skin cell disaggregation using complex coordinated motion}
        \resumeItem{Project Bioreactor: designed and implemented multi-node CANopen network for mammalian cell bioreactor consisting of closed-loop heaters, vector-controlled DC brushless motors, stepper driven peristaltic pumps and other embedded sensors}     
        \resumeItem{Project BEST: assisted team in bug squashing, software tool validation, writing test protocols, managing verification matrix, and design FMEA effort during IEC-62304 design verification of low-power wearable medical device }
        \resumeItemListEnd
    \resumeSubheading
      {San Diego State University}{Jan 2018 -- May 2019}
      {Graduate Research Assistant}{San Diego, CA}
      \resumeItemListStart
        \resumeItem{Worked with Professor Yusuf Ozturk and other professors in the Smart Health Institute to design and develop a Bluetooth motion sensor for body kinematics on Nordic nrf52840 SoC}
      \resumeItemListEnd
  \resumeSubHeadingListEnd

%-----------PROGRAMMING SKILLS-----------
\section{Technical Skills}
 \begin{itemize}[leftmargin=0.15in, label={}]
    \small{\item{
     \textbf{Languages}{: C, Python, C++, C\#, Assembly (ARM-v7-A/R/M), Make, Rust, Protobuf} \\
     \textbf{SoCs}{: STM32H7, STM32F4, STM32L0, nrf52, Zynq Ultrascale, Zynq 7000, PIC32, AVR} \\
     \textbf{Embedded Concepts}{: BLE, TCP, I2C, SPI, UART, ADC, CAN, Ethernet, FIR/IIR, PID, BLDC, Steppers } \\
     \textbf{Technologies}{: GCC, FreeRTOS, ThreadX, NetX, SEGGER J-Link, nanopb, GDB, Docker, wolfSSL, Ceedling, Github Actions} \\
     \textbf{Debug Tools}{: Logic Analyzers, Oscilloscopes, Power Analyzers, }
    }}
 \end{itemize}

%-------------------------------------------
\end{document}
